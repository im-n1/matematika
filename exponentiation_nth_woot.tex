\documentclass[a4paper,11pt]{article}
\usepackage{polyglossia}
\usepackage{helvet}
\usepackage{microtype}
\usepackage{fancyhdr}
\usepackage{paralist}
\usepackage{amsmath}
\usepackage{graphicx}
\usepackage{subcaption}
\usepackage{xcolor}
\usepackage{parskip}
\usepackage{titlesec}

\setdefaultlanguage{czech}

\author{Pavel Hrdina}

\titleformat{\section}{\Large}{}{0em}{}[\titlerule]

\title{Mocniny a odmocniny}

\begin{document}
\maketitle

\section{Mocniny}
\begin{equation*}
    \scalebox{1.6}{$
        z^n 
    $}
\end{equation*}

\begin{description}
    \item{\textbf{násobení}}\\
        exponenty se sčítají
        \begin{equation*}
            x^2 * x^4 = x^6
        \end{equation*}
    \item{\textbf{dělení}}\\
        exponenty se odčítají
        \begin{equation*}
            \frac{x^4}{x^2} = x^2
        \end{equation*}
    \item{\textbf{mocnění závorky}}\\
        mocnitelé jednotlivých prvků závorky se násobí mocnitelem závorky
        \begin{equation*}
            (4xy^2)^2 = 4^2x^2y^4
        \end{equation*}
        \begin{equation*}
            2x(3x^2y^3)^2 = 2x*3^2*x^4*y^6
        \end{equation*}
    \item{\textbf{mocnění záporným mocnitelem}}\\
        mocněnec se násobí mocnitelem a převede do zlomku pod 1
        \begin{equation*}
            2^{-2} = \frac{1}{4}
        \end{equation*}
    \item{\textbf{mocnění závorky se zlomkem záporným mocnitelem}}\\
        zlomek se převrátí a mocnitel se stane kladný
        \begin{equation*}
            \left(\frac{2x}{y^2}\right)^{-3} = \left(\frac{y^2}{2x}\right)^3
        \end{equation*}
    \item{\textbf{umocňování zlomku}}\\
        u zlomku je vykásoben čitatel i jmenovatel
        \begin{equation*}
            \left(\frac{x}{y}\right)^2 = \frac{a^2}{y^2}
        \end{equation*}
    \item{\textbf{umocňování na zlomek}}\\
        mocnina je dle čitatele a odmocnina dle jmenovatele\\
        \begin{equation*}
            27^\frac{2}{3} = \sqrt[3]{27^2}
        \end{equation*}
\end{description}

\section{Odmocniny}
\begin{equation*}
    \scalebox{1.6}{$
        \sqrt[n]{x}
    $}
\end{equation*}

\begin{description}
    \item{\textbf{odmocnění mocniny}}\\
        mocninu mocněnce lze přemístit nad odmocninu
        \begin{equation*}
            \sqrt[n]{x^k} = \sqrt[n]{x}^k
        \end{equation*}
\end{description}

\end{document}
