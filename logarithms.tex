\documentclass[a4paper,11pt]{article}
\usepackage{polyglossia}
\usepackage{helvet}
\usepackage{microtype}
\usepackage{fancyhdr}
\usepackage{paralist}
\usepackage{amsmath}
\usepackage{graphicx}
\usepackage{subcaption}
\usepackage{xcolor}
\usepackage{parskip}
\usepackage{titlesec}

\setdefaultlanguage{czech}

\author{Pavel Hrdina}

\titleformat{\section}{\Large}{}{0em}{}[\titlerule]

\title{Logaritmy}

\begin{document}
\maketitle

\section{Obecné}
\begin{equation*}
    \scalebox{1.6}{$
        y = \log_{a} x
    $}
\end{equation*}
\begin{description}
    \item[\boldmath{$y$}] - výsledný exponent mocniny, který hledáme (mocnitel)
    \item[\boldmath{$a$}] - základ (mocněnec)
    \item[\boldmath{$x$}] - výdledek mocnění $mocnenec^{mocnitel}$
\end{description}

platí:

\begin{equation*}
    \scalebox{1.6}{$y = \log_{a} x$}
\end{equation*}
\begin{center}
    je rovno
\end{center}
\begin{equation*}
    \scalebox{1.6}{$x = a^y$}
\end{equation*}

\section{Vlastnosti logaritmů}
\begin{description}
    \item{\boldmath{$\log_b b = 1$}}\\
        logaritmus toho samého čísla je vždy 1\\
        např: $\log_{10} 10 = 1$ protože $10^1 = 10$
    \item{\boldmath{$\log_b 1 = 0$}}\\
        logaritmus jedné při jakémkoli základu je vždy 0\\
        např: $\log_5 1 = 0$ protože $5^0 = 1$
    \item{\textbf{násobení}}\\
        logaritmus součinu je součet logaritnů jednotlivých činitelů
        \begin{equation*}
            \log_b ac = \log_b a + \log_b c
        \end{equation*}
    \item{\textbf{dělení}}\\
        logaritmus podílu je rozdíl logaritmů čitatele a jmenovatele
        \begin{equation*}
            \log_b \frac a c = \log_b a - \log_b c
        \end{equation*}
\end{description}

{
    \definecolor{red}{RGB}{199, 68, 64}
    \begin{figure}[ht]
        \definecolor{violet}{RGB}{96, 66, 166}
        \includegraphics[width=\textwidth]{./imgs/logaritmy/log_comparison.png}
        \caption{
            porovnání \textcolor{red}{$\log_2 x$}, \textcolor{violet}{$log_3 x$}, $log_4 x$\\
        }
    \end{figure}

    Základ logaritmu udává kolika-násobek $x$ je další stupeň na $y$. Jinými slovy
    základ logaritmu (na obrázku jako $\log$ 2, 3 a 4) udává násobek mezi jednotlivými
    stupni $x$. Kupříkladu abych se pro \textcolor{red}{$\log_3 x$} na ose $y$ mohl
    posunout z 1 (rovno 3 na ose $x$) na 2, musím na ose $x$ popojít 3-krát aktuální
    hodnotu - $3*3 = 9$.
}

\end{document}
