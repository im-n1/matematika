\documentclass[a4paper,11pt]{article}
\usepackage{polyglossia}
\usepackage{helvet}
\usepackage{microtype}
\usepackage{fancyhdr}
\usepackage{paralist}
\usepackage{amsmath}
\usepackage{graphicx}
\usepackage{subcaption}
\usepackage{xcolor}
\usepackage{parskip}
\usepackage{titlesec}

\setdefaultlanguage{czech}

\author{Pavel Hrdina}

\titleformat{\section}{\Large}{}{0em}{}[\titlerule]

\title{Čísla}

\begin{document}
\maketitle

\section{Obecné}
\begin{description}
    \item{\textbf{Reálné číslo ($R$)}}\\
        lze zapsat pomocí konečného či nekonečného desetinného
        rozvoje\\
        např: 1, -3, 0.1, -0.1, e, 0, -2$\pi$\\
        platí: ${N \subset Z \subset Q \subset R}$

        \begin{description}
            \item{\textbf{Racionální číslo ($Q$)}}\\
                lze vyjádřit jako zlomek vyjíma nuly ve jmenovateli\\
                např: $\frac{1}{2}$, $\frac{a}{b}$\\
                platí: $N \subset Z \subset Q$, 

                \begin{description}
                    \item{\textbf{Celé číslo ($Z$)}}\\
                        kladné i záporné číslo včetně nuly\\
                        např: -1, 0, 1\\
                        platí: $N \subset Z$

                        \begin{description}
                            \item{\textbf{Přirozené číslo ($N$)}}\\
                                celé číslo vyjíma nuly\\
                                např: 1, 2, 3\\
                        \end{description}

                \end{description}

        \item{\textbf{Iracionální číslo ($I$)}}\\
            reálné číslo, které není racionálním číslem - nelze
            zapsat ve tvaru zlomku\\
            např: $\pi$, $e$, $\sqrt{2}$\\
            platí: $I \cap Q = 0$
        \end{description}

\end{description}
\begin{figure}[ht]
    \includegraphics[width=\textwidth]{imgs/zaklady/number_types_scheme.png}
    \caption{schéma množin čísel}
\end{figure}

\end{document}
